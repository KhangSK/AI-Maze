\documentclass[a4paper]{article}
\usepackage{vntex}
\author{Lăng Hoàng Long, Hoàng Gia Khang}
\date{Tháng 5, 2020}

\usepackage[top=3cm, bottom=4cm, left=3cm, right=3cm]{geometry}
\usepackage{hyperref}
\hypersetup{
	colorlinks=true,
	linkcolor=blue,
	filecolor=blue,
	urlcolor=blue,
}
\usepackage{lastpage}
\usepackage{afterpage}
\newcommand\blankpage{
	\null
	\thispagestyle{empty}
	\addtocounter{page}{-1}
	\newpage
}
\usepackage{indentfirst}
\setlength{\parindent}{0.6cm}

\usepackage{graphicx}
\usepackage{float}
\usepackage{listings}
\usepackage{xcolor}
\lstset{
	basicstyle=\footnotesize\ttfamily,
	backgroundcolor=\color{gray!10},
	frame=tblr,
	tabsize=4,
	numbers=left,
	commentstyle=\color{gray},
	keywordstyle=\color{blue},
	stringstyle=\color{violet},
	breaklines=true,
	showstringspaces=false,
	captionpos=b
}

\newcommand{\Image}[3]{
	\begin{figure}[H]
		\begin{center}
			\includegraphics[scale=#2]{#1}
			\caption{#3}
		\end{center}
	\end{figure}
}

\usepackage{fancyhdr}
\setlength{\headheight}{40pt}
\pagestyle{fancy}
\fancyhead{}
\fancyhead[L]{
	\begin{tabular}{rl}
		\begin{picture}(24,14)(0,0)
		\put(0,-8){\includegraphics[width=8mm, height=8mm]{Image/BKHCM.png}}
		\end{picture}
		\begin{tabular}{l}
			\textbf{\texttt{Trường đại học Bách Khoa TPHCM}}\\
			\textbf{\texttt{Khoa Khoa học và Kỹ thuật Máy tính}}\\
		\end{tabular}
	\end{tabular}
}
\fancyhead[R]{
	\begin{tabular}{l}
		\tiny \bf \\
		\tiny \bf
	\end{tabular}
}
\fancyfoot{}
\fancyfoot[L]{\scriptsize \ttfamily Học kì 192 - Năm học 2019-2020}
\fancyfoot[R]{\scriptsize \ttfamily Trang {\thepage}/\pageref{LastPage}}



\begin{document}

	\begin{titlepage}
		\begin{center}
			\vspace*{2cm}
			\large{TRƯỜNG ĐẠI HỌC BÁCH KHOA TPHCM}\\
			\large{\textbf{KHOA KHOA HỌC VÀ KỸ THUẬT MÁY TÍNH}}\\
			\verb|------------------------------------|\\

			\begin{figure}[H]
				\centering
				\includegraphics[scale=0.3]{Image/BKHCM.png}
			\end{figure}

			\begin{tabular}{c}
				\multicolumn{1}{c}{\textbf{\LARGE{Nhập môn trí tuệ nhân tạo}}}\\
				\\ \hline \\
				\multicolumn{1}{l}{\textbf{\Large{Phần mở rộng KSTN}}}\\
				\\
				\textbf{\Huge{Hiện thực bài toán}}\\
				\textbf{\Huge{tìm đường đi trong mê cung}}\\
				\\ \hline \\
			\end{tabular}

			\begin{table}[h]
				\begin{tabular}{rlr}
					\hspace{3cm}
					\textbf{GVHD}:      & Nguyễn Hồ Mẫn Rạng &         \\
					\textbf{Sinh viên}: & Lăng Hoàng Long    & 1812879 \\
					                    & Hoàng Gia Khang    & 1812535 \\
				\end{tabular}
			\end{table}

			\vspace{3cm}

			\textit{Thành phố Hồ Chí Minh, tháng 5 - 2020}
		\end{center}
	\end{titlepage}



	\section{Yêu cầu đề bài}
	Hiện thực bài toán tìm đường đi trong mê cung:
	\begin{enumerate}
		\item Hiển thị giao diện trực quan mê cung
		\item Có chức năng thay đổi mê cung thông qua file cấu hình (kích thước mê cung tối đa 10x10)
		\item Có chức năng điều chỉnh tốc độ chạy
		\item Phần hiện thực cần bao gồm các giải thuật sau:
			\begin{itemize}
				\item Depth First Search (DFS)
				\item Breadth First Search (BFS)
				\item Uniform Cost Search (UCS)
				\item Best First Search (Greedy Search)
				\item A-Star
			\end{itemize}
		\item Ngôn ngữ: Python
	\end{enumerate}

	\section{Hiện thực}
	\subsection{Mã nguồn}
	Link GitHub của phần hiện thực: \url{https://github.com/khangsk/AI}\\

	Bên trong repository có chứa các file như sau:
	\begin{center}
		\begin{tabular}{|p{3cm}|p{9cm}|}
			\hline
			\textbf{Tên file} & \textbf{Chức năng}\\
			\hline
			\texttt{util.py} & Hiện thực các cấu trúc dữ liệu cơ bản (Queue, Stack, ...) và một số hàm (manhattanDistance)\\
			\hline
			\texttt{AI.py} & Chứa phần đồ họa và các thuật toán tìm kiếm\\
			\hline
			\texttt{Report.tex} & Mã nguồn \LaTeX của báo cáo này\\
			\hline
			Các file có đuôi \texttt{.txt} & file cấu hình mê cung\\
			\hline
		\end{tabular}
	\end{center}

	\subsection{Chi tiết hiện thực}

	\paragraph{Về đồ họa}\par
	Chúng em sử dụng thư viện \href{turtle}{https://docs.python.org/3/library/turtle.html} có sẵn của Python để xây dựng phần đồ họa.
	\Image{Image/Demo.png}{1}{Đồ họa sử dụng thư viện turtle \label{Figure:Demo}}

	\paragraph{Về cách cấu hình}\par
	Chương trình cho phép cấu hình thông qua file text, với định dạng như sau:
	\begin{center}
		\begin{tabular}{|p{3cm}|p{9cm}|}
			\hline
			\textbf{Ký tự} & \textbf{Chức năng}\\
			\hline
			<space> & Thể hiện một ô trống có thể đi qua được\\
			\hline
			\texttt{\%} & Thể hiện ô có tường\\
			\hline
			\texttt{S} & Vị trí xuất phát\\
			\hline
			\texttt{G} & Vị trí ô đích\\
			\hline
		\end{tabular}
	\end{center}

	Đây là file cấu hình tương ứng với hình \ref{Figure:Demo}:\\
	\lstinputlisting[title=File cấu hình mẫu]{test.txt}

	\paragraph{Điều chỉnh tốc độ}\par
	Chương trình cho phép điều chỉnh tốc độ chạy bằng tham số.

	\paragraph{Các thuật toán}\par
	Cả 5 thuật toán được yêu cầu đều đã được hiện thực trong file \texttt{AI.py}. Có thể lựa chọn thuật toán được sử dụng bằng tham số.
	\Image{Image/Algo.png}{0.8}{Các thuật toán đã được hiện thực}

	\paragraph{Ngôn ngữ}\par
	Chúng em viết chương trình này bằng Python 100\%.



	\section{Sử dụng}
	Dòng lệnh chạy chương trình có dạng như sau:\\
	\begin{center}
	\texttt{\$ python AI.py <config\_file> <algorithm> <delay>}
	\end{center}

	trong đó:
	\begin{center}
		\begin{tabular}{p{3cm}p{9cm}}
			\texttt{<config\_file>} & Tên file cấu hình\\
			\texttt{<algorithm>} & Thuật toán được sử dụng. Chỉ có 5 lựa chọn là \texttt{DFS}, \texttt{BFS}, \texttt{BestFS}, \texttt{UCS} và \texttt{AStar} tương ứng với 5 thuật toán được yêu cầu\\
			\texttt{<delay>} & Thời gian chờ giữa hai khung hình (số càng nhỏ chạy càng nhanh)\\
		\end{tabular}
	\end{center}

\end{document}
